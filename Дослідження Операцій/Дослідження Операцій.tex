% These lines tell TeXShop to typeset with xelatex, and to open and 
% save the source with Unicode encoding.

%!TEX TS-program = xelatex
%!TEX encoding = UTF-8 Unicode

\documentclass[12pt]{book}
\usepackage{xltxtra}
\usepackage{makecell}
\usepackage{geometry}
\usepackage{amsmath}
\geometry{a4paper}
\renewcommand{\chaptername}{Тема}
\setmainfont[Mapping=tex-text]{Times New Roman}
\begin{document}
\tableofcontents
\chapter{Вступ}
Операція - сукупність дій спрямованих на досягнення визначеної мети.
Дослідження операцій - наука яка займається дослідженням реальних процесів-операцій, виробленням рекомендацій про прийняття рішень.
Кроки при розв'язуванні задач:
\begin{enumerate}
\item Постановка задачі.
\item Побудова моделі.
\item Відшукання розв'язу.
\item Перевірка моделі та оцінка результату.
\item Впровадження розв'язку та контроль достовірності.
\end{enumerate}
В залежності від задачі і побудованої моделі використовують різні методи, в той час, як один метод можна застосувати до різних задач. Зараз багато методів займаються лінійним програмуванням, теорією ігор і тд.

\chapter{Лінійні моделі та дослідження операцій}
У випадку, якщо вихідна задача не є лінійною, то на першому етапі дослідження її або вважають лінійною, або всі залежності замінюються лінійними. Для побудови методу використовується апарат лінійного програмування.
\section{Транспортна задача}
Нехай є m пунктів виробництва…

Транспортну задачу зручно зображати таблицею:\\

\begin{tabular}{ | c | c | c | c | c | }
\hline
\diaghead(4,3){easterr}{$c_{1 2}$}{$x_{1 2}$} & \diaghead(4,3){easterr}{$c_{1 2}$}{$x_{1 2}$} & \thead{\vdots} & \diaghead(4,3){easterr}{$c_{1 n}$}{$x_{1 n}$} & \thead{$a_1$} \\
\hline
\diaghead(4,3){easterr}{$c_{2 1}$}{$x_{2 1}$} & \diaghead(4,3){easterr}{$c_{2 2}$}{$x_{2 2}$} & \thead{\vdots} & \diaghead(4,3){easterr}{$c_{2 n}$}{$x_{2 n}$} & \thead{$a_2$} \\
\hline
 \thead{$\cdots$} & \thead{$\cdots$} & \thead{$\ddots$} & \thead{$\cdots$} & \thead{$\cdots$} \\
\hline
\diaghead(4,3){easterr}{$c_{m 1}$}{$x_{m 1}$} & \diaghead(4,3){easterr}{$c_{m 2}$}{$x_{m 2}$} & \thead{\vdots} & \diaghead(4,3){easterr}{$c_{m n}$}{$x_{m n}$} & \thead{$a_m$} \\
\hline
\thead{$b_1$} & \thead{$b_2$} & \thead{\vdots} & \thead{$b_n$} & \thead{} \\
\hline
\end{tabular}
\\

\begin{tabular}{ @{\hspace{1.4em}}l l }
$
\setlength{\arraycolsep}{0.27em}
\begin{array}{ccccccccccccc}
A_{1 1} & A_{1 2} & \dots & A_{1 n} & A_{2 1} & A_{2 2} & \dots & A_{2 n} & \dots & A_{m 1} & A_{m 2} & \dots & A_{m n} 
\end{array}$ &  \\
\multicolumn{2}{l}{
$\left(
 \begin{array}{ccccccccccccc}
1 & 1 & \dots & 1 & 0 & 0 & \dots & 0 & \dots & 0 & 0 & \dots & 0 \\
0 & 0 & \dots & 0 & 1 & 1 & \dots & 1 & \dots & 0 & 0 & \dots & 0 \\
\dots & \dots & \dots & \dots & \dots & \dots & \dots & \dots & \dots & \dots & \dots & \dots & \dots \\
0 & 0 & \dots & 0 & 0 & 0 & \dots & 0 & \dots & 1 & 1 & \dots & 1 \\
1 & 0 & \dots & 0 & 1 & 0 & \dots & 0 & \dots & 1 & 0 & \dots & 0 \\
\dots & \dots & \dots & \dots & \dots & \dots & \dots & \dots & \dots & \dots & \dots & \dots & \dots \\
0 & 0 & \dots & 1 & 0 & 0 & \dots & 1 & \dots & 0 & 0 & \dots & 1
\end{array}\right)
\left(\begin{array}{c}
x_{1 1} \\
x_{1 2} \\
\dots \\
x_{1 n} \\
x_{m 1} \\
\dots \\
x_{m n}
\end{array}
\right)
=
\left(\begin{array}{c}
a_1 \\
a_2 \\
\dots \\
a_m \\
b_1 \\
\dots \\
b_n
\end{array}
\right)$}
\end{tabular}

\begin{tabular}{ @{\hspace{1em}}l l }
$
\setlength{\arraycolsep}{0.23em}
\begin{array}{cccccccc}
A_{1 n} & A_{2 n} & \dots & A_{m n} &  A_{1 1} & A_{1 2} & \dots & A_{1 n-1} 
\end{array}$ &  \\
\multicolumn{2}{l}{
$\left|
 \begin{array}{cccccccc}
1 & 0 & \dots & 0 & 1 & 1 & \dots & 1 \\
0 & 1 & \dots & 0 & 0 & 0 & \dots & 0 \\
\dots & \dots & \dots & \dots & \dots & \dots & \dots & \dots \\
0 & 0 & \dots & 1 & 0 & 0 & \dots & 0 \\
0 & 0 & \dots & 0 & 1 & 0 & \dots & 0 \\
0 & 0 & \dots & 0 & 0 & 1 & \dots & 0 \\
\dots & \dots & \dots & \dots & \dots & \dots & \dots & \dots \\
0 & 0 & \dots & 0 & 0 & 0 & \dots & 1 
\end{array}\right|$}
\end{tabular}

\end{document}