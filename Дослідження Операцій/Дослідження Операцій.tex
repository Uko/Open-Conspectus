% These lines tell TeXShop to typeset with xelatex, and to open and 
% save the source with Unicode encoding.

%!TEX TS-program = xelatex
%!TEX encoding = UTF-8 Unicode

\documentclass[12pt]{book}
\usepackage{xltxtra}
\usepackage{makecell}
\usepackage{geometry}
\usepackage{amsmath}
\geometry{a4paper}
\renewcommand{\chaptername}{Тема}
\setmainfont[Mapping=tex-text]{Times New Roman}
\begin{document}
\tableofcontents
\chapter{Вступ}
Операція - сукупність дій, спрямованих на досягнення визначеної мети.
Дослідження операцій - наука, що займається дослідженням реальних процесів-операцій, виробленням рекомендацій про прийняття рішень.\\
Кроки при розв'язуванні задач:
\begin{enumerate}
\item Постановка задачі.
\item Побудова моделі.
\item Відшукання розв'язку.
\item Перевірка моделі та оцінка результату.
\item Впровадження розв'язку та контроль його достовірності.
\end{enumerate}
В залежності від задачі і побудованої моделі використовують різні методи, в той час, як один метод можна застосувати до різних задач. Зараз багато методів займаються лінійним програмуванням, теорією ігор і тд.

\chapter{Лінійні моделі дослідження операцій}
У випадку, якщо вихідна задача не є лінійною, то на першому етапі дослідження її або вважають лінійною, або всі залежності замінюються лінійними. Для побудови методу використовується апарат лінійного програмування.
\section{Транспортна задача}
Нехай маємо m пунктів виробництва однорідного продукту з потужностями відповідно $a_i$, $ i =\overline{1, m}$. Маємо n пунктів споживання $b_j$, $ j =\overline{1, n}$. Задається матриця перевезень С =\{ $c_{ij}$ \}, де $c_{ij}$ - вартість перевезення одиниці продукту з i-того пункту виробництва в j-ий пункт споживання. Потрібно знайти такий набір $x_{ij}$ $\geq$ 0, $ i = \overline{1, m}, j = \overline{1, n}$, де  $x_{ij}$ - кількість одиниць продукту, яка перевозиться з і-ого пункту виробництва в j-ий пункт споживання, щоб виконувались наступні умови: 
\begin{enumerate}
\item  \begin{equation}  \sum_{j=1}^n x_{ij} = a_i, i = \overline{1, m}. \end{equation}
\item   \begin{equation} \sum_{i=1}^m x_{ij} = b_j, j = \overline{1, n}.   \end{equation}
\item  \begin{equation} \sum_{i=1}^m \sum_{j=1}^n c_{ij} x_{ij} \to \min  \end{equation}
\end{enumerate}
Невідємний набір $x_{ij}$, який задовольняє (2.1), (2.2), називається планом задачі або допустимим розвязком. Той із планів, який надає мінімум в (2.3), називається оптимальним планом або розвязком транспортної задачі.
Зауважимо, що транспортна задача, поставлена в такій формі, називається транспортною задачею за критерієм вартості.
Умова \begin{equation}  \sum_{i=1}^n a_i = \sum_{j=1}^m b_j  \end{equation} називається умовою балансу мас.\\
Транспортну задачу зручно зображати таблицею:\\

\begin{tabular}{ | c | c | c | c | c | }
\hline
\diaghead(4,3){easterr}{$c_{1 2}$}{$x_{1 2}$} & \diaghead(4,3){easterr}{$c_{1 2}$}{$x_{1 2}$} & \thead{\vdots} & \diaghead(4,3){easterr}{$c_{1 n}$}{$x_{1 n}$} & \thead{$a_1$} \\
\hline
\diaghead(4,3){easterr}{$c_{2 1}$}{$x_{2 1}$} & \diaghead(4,3){easterr}{$c_{2 2}$}{$x_{2 2}$} & \thead{\vdots} & \diaghead(4,3){easterr}{$c_{2 n}$}{$x_{2 n}$} & \thead{$a_2$} \\
\hline
 \thead{$\cdots$} & \thead{$\cdots$} & \thead{$\ddots$} & \thead{$\cdots$} & \thead{$\cdots$} \\
\hline
\diaghead(4,3){easterr}{$c_{m 1}$}{$x_{m 1}$} & \diaghead(4,3){easterr}{$c_{m 2}$}{$x_{m 2}$} & \thead{\vdots} & \diaghead(4,3){easterr}{$c_{m n}$}{$x_{m n}$} & \thead{$a_m$} \\
\hline
\thead{$b_1$} & \thead{$b_2$} & \thead{\vdots} & \thead{$b_n$} & \thead{} \\
\hline
\end{tabular}
\\

\begin{tabular}{ @{\hspace{1.4em}}l l }
$
\setlength{\arraycolsep}{0.27em}
\begin{array}{ccccccccccccc}
A_{1 1} & A_{1 2} & \dots & A_{1 n} & A_{2 1} & A_{2 2} & \dots & A_{2 n} & \dots & A_{m 1} & A_{m 2} & \dots & A_{m n} 
\end{array}$ &  \\
\multicolumn{2}{l}{
$\left(
 \begin{array}{ccccccccccccc}
1 & 1 & \dots & 1 & 0 & 0 & \dots & 0 & \dots & 0 & 0 & \dots & 0 \\
0 & 0 & \dots & 0 & 1 & 1 & \dots & 1 & \dots & 0 & 0 & \dots & 0 \\
\dots & \dots & \dots & \dots & \dots & \dots & \dots & \dots & \dots & \dots & \dots & \dots & \dots \\
0 & 0 & \dots & 0 & 0 & 0 & \dots & 0 & \dots & 1 & 1 & \dots & 1 \\
1 & 0 & \dots & 0 & 1 & 0 & \dots & 0 & \dots & 1 & 0 & \dots & 0 \\
\dots & \dots & \dots & \dots & \dots & \dots & \dots & \dots & \dots & \dots & \dots & \dots & \dots \\
0 & 0 & \dots & 1 & 0 & 0 & \dots & 1 & \dots & 0 & 0 & \dots & 1
\end{array}\right)
\left(\begin{array}{c}
x_{1 1} \\
x_{1 2} \\
\dots \\
x_{1 n} \\
x_{m 1} \\
\dots \\
x_{m n}
\end{array}
\right)
=
\left(\begin{array}{c}
a_1 \\
a_2 \\
\dots \\
a_m \\
b_1 \\
\dots \\
b_n
\end{array}
\right)$}
\end{tabular}

\begin{tabular}{ @{\hspace{1em}}l l }
$
\setlength{\arraycolsep}{0.23em}
\begin{array}{cccccccc}
A_{1 n} & A_{2 n} & \dots & A_{m n} &  A_{1 1} & A_{1 2} & \dots & A_{1 n-1} 
\end{array}$ &  \\
\multicolumn{2}{l}{
$\left|
 \begin{array}{cccccccc}
1 & 0 & \dots & 0 & 1 & 1 & \dots & 1 \\
0 & 1 & \dots & 0 & 0 & 0 & \dots & 0 \\
\dots & \dots & \dots & \dots & \dots & \dots & \dots & \dots \\
0 & 0 & \dots & 1 & 0 & 0 & \dots & 0 \\
0 & 0 & \dots & 0 & 1 & 0 & \dots & 0 \\
0 & 0 & \dots & 0 & 0 & 1 & \dots & 0 \\
\dots & \dots & \dots & \dots & \dots & \dots & \dots & \dots \\
0 & 0 & \dots & 0 & 0 & 0 & \dots & 1 
\end{array}\right|$}
\end{tabular}

\\$\Rightarrow$
\\
\subsection{Теорема  2}\\
Для розвязності транспортної задачі необхідно і достатньо, щоб виконувалась умова балансу мас $\sum_{i=1}^m a_i = \sum_{j=1}^n b_j$ .
\\
$\mathbf {Доведення:}$\\
(Необхідність)\\ 
 Нехай $ x_{ij}^*, i = \overline{1, m}, j = \overline{1, n}$ , - розвязок транспортної задачі. \\
 Оскільки  $\sum_{j=1}^n x_{ij}^* = a_i,  i = \overline{1, m};  \sum_{i=1}^m  x_{ij}^* = b_j ,  j = \overline{1, n}$ , то отримаємо
 $\sum_{i=1}^n a_i = \sum_{j=1}^m b_j$.
\\
(Достатність)
\\
Нехай виконується умова (2.4), покладемо  $x_{ij} = \frac{a_ib_j}{\sum a_i}$. Сумуючи це співвідношення по j, отримаємо:\\
$\sum_{j=1}^n x_{ij} = \sum_{j=1}^n \frac{a_ib_j}{\sum a_i} = a_i,  i = \overline{1, m}.$
\\ $\sum_{j=1}^m x_{ij} = \sum_{i=1}^m \frac{a_ib_j}{\sum a_i} = b_j,  j = \overline{1, n}.$
\\
\section{Властивості опорних планів транспортної задачі}
План $\{x_{ij}\}_{m,n}$ транспортної задачі називають опорним планом, якщо вектори $A_{ij}$, що відповідають додатним компонентам плану лінійно незалежні.\\
\\
Оскільки із Т1 $\Rightarrow$ ранг матриці обм. транспортної задачі $r=m+n-1$, то додатних компонент опорний план може мати не більше ніж $m+n-1$.
\begin{itemize}
\item Опорний план, який має рівно $m+n-1$ додатних компонент - невироджений.
\item Опорний план, який має менше $m+n-1$ додатних компонент - вироджений.
\end{itemize}
Базисом ОП називається довільна система із $m+n-1$ лінійно незалежних векторів $A_{ij}$, яка містить усі вектори $A_{ij}$, що відповідають додатним компонентам плану.\\
Поставимо взаємовідповідність між клітинами транспонованої таблиці і векторами $A_{ij}$.\\
Кожній клітині $(i,j){\leftrightarrow}A_{ij}$.\\
Набір клітин $$(i_1,j_1),(i_1,j_2),(i_2,j_2),\dots,(i_s,j_1)$$або$$(i_1,j_1),(i_2,j_1),(i_2,j_2),\dots,(i_1,j_2)$$
називають ланцюжком. Звідси видно, що два сусідні елементи ланцюжка лежать або в одному рядку або в одному стовпчику.\\
\\
Зауважимо, що кількість елементів замкненого ланцюжка завжди парна.\\
\\
\dots десь тут була табличка \dots\\
\\
Нехай $P$ - довільна система векторів $A_{ij}$ умов. транспортної задачі, $I$ - множина пар індексів $(i,j)$, які відповідають векторам $A_{ij}{\in}P$.
\subsection{Теорема 3}
Для того, щоб система векторів $P$ була лінійно незалежною необхідно і достатньо, щоб із елементів множини $I$ неможна було скласти замкнений ланцюжок.\\
Доведення.\\
Необхідність. $P$ - лінійно незалежна система, покажем, що неможна замкнути ланцюжок. Від супротивного. Припустимо, що із елементів множини $I$ можна скласти замкнений ланцюжок: $$(i_1,j_1), (i_1,j_2), (i_2,j_2), \dots, (i_s,j_1).$$ Звідси випливає, враховуючи вигляд векторів $A_i$, $$A_{{i_1},{j_1}}-A_{{i_1},{j_2}}+A_{{i_2},{j_2}}-\dots-A_{{i_s},{j_1}}=0,$$ а тому система векторів лінійно залежна, що суперечить вхідній умові.\\
Достатність. Припустим, що замкнений ланцюжок не скласти. Покажем, що система векторів лінійно незалежна. Від супротивного. Нехай вектори лінійно незалежні. Звідси випливає, що існує $\alpha_{ij}:$ $$\sum_{(i,j){\in}I}\alpha_{ij}A_{ij} = 0.$$
Нехай $\alpha_{{i_1},{j_1}} \neq 0$, тоді: $$\sum_{(i,j) \in I}\alpha_{ij}A_{ij} = -\alpha_{{i_1}{j_1}}A_{{i_1}{j_1}};$$$$I_1 = I\setminus\{(i_1,j_1)\}.$$
Компонента $I_1$ вектора в правій частині не рівна нулю, тому в лівій частині існує принаймі один вектор $A_{{i_1}{j_2}}: \alpha_{{i_1}{j_2}}\neq0$, тоді $$\sum_{(i,j){\in}I}\alpha_{ij}A_{ij} = -\alpha_{{i_1}{j_1}}A_{{i_1}{j_1}}-\alpha_{{i_1}{j_2}}A_{{i_1}{j_2}}.$$
Оскільки $j_1 \neq j_2$ і $m + j_2$ компонента парвої частини не рівна нулю, то знайдеться принаймі один вектор $A_{{i_2}{j_2}}: \alpha_{{2_1}{j_2}}\neq0$, тоді $$\sum_{(i,j){\in}I}\alpha_{ij}A_{ij} = -\alpha_{{i_1}{j_1}}A_{{i_1}{j_1}}-\alpha_{{i_1}{j_2}}A_{{i_1}{j_2}}-\alpha_{{i_2}{j_2}}A_{{i_2}{j_2}}.$$
і т.д.\\
Цей процес скінченний, оскільки всі вектори в лівій частині різні, то врезультаті приходимо до: $$-\alpha_{{i_1}{j_1}}A_{{i_1}{j_1}}-\alpha_{{i_1}{j_2}}A_{{i_1}{j_2}}-\dots-alpha_{{i_k}{j_k}}A_{{i_k}{j_k}}, i_k=i_s, 1{\leq}s{\leq}k-1;$$$$- \alpha_{{i_1}{j_1}}A_{{i_1}{j_1}}-\alpha_{{i_1}{j_2}}A_{{i_1}{j_2}}-\dots-\alpha_{{i_k}{j_{k+1}}}A_{{i_k}{j_{k+1}}}, i_{k+1}=i_l, l{\leq}1{\leq}k-1.$$

\end{document}