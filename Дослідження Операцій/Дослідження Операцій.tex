% These lines tell TeXShop to typeset with xelatex, and to open and 
% save the source with Unicode encoding.

%!TEX TS-program = xelatex
%!TEX encoding = UTF-8 Unicode

\documentclass[12pt]{book}
\usepackage{xltxtra}
\usepackage{slashbox}
\usepackage{geometry}
\geometry{a4paper}
\renewcommand{\chaptername}{Тема}
\setmainfont[Mapping=tex-text]{Times New Roman}
\begin{document}
\tableofcontents
\chapter{Вступ}
Операція - сукупність дій спрямованих на досягнення визначеної мети.
Дослідження операцій - наука яка займається дослідженням реальних процесів-операцій, виробленням рекомендацій про прийняття рішень.
Кроки при розв'язуванні задач:
\begin{enumerate}
\item Постановка задачі.
\item Побудова моделі.
\item Відшукання розв'язу.
\item Перевірка моделі та оцінка результату.
\item Впровадження розв'язку та контроль достовірності.
\end{enumerate}
В залежності від задачі і побудованої моделі використовують різні методи, в той час, як один метод можна застосувати до різних задач. Зараз багато методів займаються лінійним програмуванням, теорією ігор і тд.

\chapter{Лінійні моделі дослідження операцій}
У випадку, якщо вихідна задача не є лінійною, то на першому етапі дослідження її або вважають лінійною, або всі залежності замінюються лінійними. Для побудови методу використовується апарат лінійного програмування.
\section{Транспортна задача}
Нехай є m пунктів виробництва…

Транспортну задачу зручно зображати таблицею:\\
\begin{tabular}{ | c | c | c | c | c | }
\hline
\slashbox{$x_{1 1}$}{$c_{1 1}$} & \slashbox{$x_{1 2}$}{$c_{1 2}$} & \vdots & \slashbox{$x_{1 n}$}{$c_{1 n}$} & $a_1$ \\
\hline
\slashbox{$x_{2 1}$}{$c_{2 1}$} & \slashbox{$x_{2 2}$}{$c_{2 2}$} & \vdots & \slashbox{$x_{2 n}$}{$c_{2 n}$} & $a_2$ \\
\hline
 $\cdots$ & $\cdots$ & $\ddots$ & $\cdots$ & $\cdots$ \\
\hline
\slashbox{$x_{m 1}$}{$c_{m 1}$} & \slashbox{$x_{m 2}$}{$c_{m 2}$} & \vdots & \slashbox{$x_{m n}$}{$c_{m n}$} & $a_m$ \\
\hline
$b_1$ & $b_2$ & \vdots & $b_n$ &  \\
\hline
\end{tabular}
\subsection{Теорема 3}
Для того, щоб система векторів P була лінійно незалежною необхідно і достатньо, щоб із елементів множини I неможна було скласти замкнений ланцюжок.
Доведення.
Необхідність. P - лінійно незалежна система, покажем, що неможна замкнути ланцюжок. Від супротивного. Припустимо, що із елементів множини I можна скласти замкнений ланцюжок: $$(i_1,j_1), (i_1,j_2), (i_2,j_2), \dots, (i_s,j_1).$$ Звідси випливає, враховуючи вигляд векторів $A_i$, $$A_{{i_1},{j_1}}-A_{{i_1},{j_2}}+A_{{i_2},{j_2}}-\dots-A_{{i_s},{j_1}}=0,$$ а тому система векторів лінійно залежна, що суперечить вхідній умові.
Достатність. Припустим, що замкнений ланцюжок не скласти. Покажем, що система векторів лінійно незалежна. Від супротивного. Нехай вектори лінійно незалежні. Звідси випливає, що існує $\alpha_{ij}:$ $$\sum_{(i,j){\in}I}\alpha_{ij}A_{ij} = 0.$$
Нехай $\alpha_{{i_1},{j_1}} \neq 0$, тоді: $$\sum_{(i,j) \in I}\alpha_{ij}A_{ij} = -\alpha_{{i_1}{j_1}}A_{{i_1}{j_1}};$$$$I_1 = I\setminus\{(i_1,j_1)\}.$$
Компонента $I_1$ вектора в правій частині не рівна нулю, тому в лівій частині існує принаймі один вектор $A_{{i_1}{j_2}}: \alpha_{{i_1}{j_2}}\neq0$, тоді $$\sum_{(i,j){\in}I}\alpha_{ij}A_{ij} = -\alpha_{{i_1}{j_1}}A_{{i_1}{j_1}}-\alpha_{{i_1}{j_2}}A_{{i_1}{j_2}}.$$
Оскільки $j_1 \neq j_2$ і $m + j_2$ компонента парвої частини не рівна нулю, то знайдеться принаймі один вектор $A_{{i_2}{j_2}}: \alpha_{{2_1}{j_2}}\neq0$, тоді $$\sum_{(i,j){\in}I}\alpha_{ij}A_{ij} = -\alpha_{{i_1}{j_1}}A_{{i_1}{j_1}}-\alpha_{{i_1}{j_2}}A_{{i_1}{j_2}}-\alpha_{{i_2}{j_2}}A_{{i_2}{j_2}}.$$
і т.д.
Цей процес скінченний, оскільки всі вектори в лівій частині різні, то врезультаті приходимо до: $$-\alpha_{{i_1}{j_1}}A_{{i_1}{j_1}}-\alpha_{{i_1}{j_2}}A_{{i_1}{j_2}}-\dots-alpha_{{i_k}{j_k}}A_{{i_k}{j_k}}, i_k=i_s, 1{\leq}s{\leq}k-1;$$$$- \alpha_{{i_1}{j_1}}A_{{i_1}{j_1}}-\alpha_{{i_1}{j_2}}A_{{i_1}{j_2}}-\dots-\alpha_{{i_k}{j_{k+1}}}A_{{i_k}{j_{k+1}}}, i_{k+1}=i_l, l{\leq}1{\leq}k-1.$$
\end{document}