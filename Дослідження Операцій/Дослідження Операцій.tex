% These lines tell TeXShop to typeset with xelatex, and to open and 
% save the source with Unicode encoding.

%!TEX TS-program = xelatex
%!TEX encoding = UTF-8 Unicode

\documentclass[12pt]{book}
\usepackage{xltxtra}
\usepackage{makecell}
\usepackage{geometry}
\geometry{a4paper}
\renewcommand{\chaptername}{Тема}
\setmainfont[Mapping=tex-text]{Times New Roman}
\begin{document}
\tableofcontents
\chapter{Вступ}
Операція - сукупність дій спрямованих на досягнення визначеної мети.
Дослідження операцій - наука яка займається дослідженням реальних процесів-операцій, виробленням рекомендацій про прийняття рішень.
Кроки при розв'язуванні задач:
\begin{enumerate}
\item Постановка задачі.
\item Побудова моделі.
\item Відшукання розв'язу.
\item Перевірка моделі та оцінка результату.
\item Впровадження розв'язку та контроль достовірності.
\end{enumerate}
В залежності від задачі і побудованої моделі використовують різні методи, в той час, як один метод можна застосувати до різних задач. Зараз багато методів займаються лінійним програмуванням, теорією ігор і тд.

\chapter{Лінійні моделі та дослідження операцій}
У випадку, якщо вихідна задача не є лінійною, то на першому етапі дослідження її або вважають лінійною, або всі залежності замінюються лінійними. Для побудови методу використовується апарат лінійного програмування.
\section{Транспортна задача}
Нехай є m пунктів виробництва…

Транспортну задачу зручно зображати таблицею:\\

\begin{tabular}{ | c | c | c | c | c | }
\hline
\diaghead(4,3){easterr}{$c_{1 2}$}{$x_{1 2}$} & \diaghead(4,3){easterr}{$c_{1 2}$}{$x_{1 2}$} & \thead{\vdots} & \diaghead(4,3){easterr}{$c_{1 n}$}{$x_{1 n}$} & \thead{$a_1$} \\
\hline
\diaghead(4,3){easterr}{$c_{2 1}$}{$x_{2 1}$} & \diaghead(4,3){easterr}{$c_{2 2}$}{$x_{2 2}$} & \thead{\vdots} & \diaghead(4,3){easterr}{$c_{2 n}$}{$x_{2 n}$} & \thead{$a_2$} \\
\hline
 \thead{$\cdots$} & \thead{$\cdots$} & \thead{$\ddots$} & \thead{$\cdots$} & \thead{$\cdots$} \\
\hline
\diaghead(4,3){easterr}{$c_{m 1}$}{$x_{m 1}$} & \diaghead(4,3){easterr}{$c_{m 2}$}{$x_{m 2}$} & \thead{\vdots} & \diaghead(4,3){easterr}{$c_{m n}$}{$x_{m n}$} & \thead{$a_m$} \\
\hline
\thead{$b_1$} & \thead{$b_2$} & \thead{\vdots} & \thead{$b_n$} & \thead{} \\
\hline
\end{tabular}
\end{document}