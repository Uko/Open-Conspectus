% These lines tell TeXShop to typeset with xelatex, and to open and 
% save the source with Unicode encoding.

%!TEX TS-program = xelatex
%!TEX encoding = UTF-8 Unicode

\documentclass[12pt]{book}
\usepackage{xltxtra}
\usepackage{slashbox}
\usepackage{geometry}
\geometry{a4paper}
\renewcommand{\chaptername}{Тема}
\setmainfont[Mapping=tex-text]{Times New Roman}
\begin{document}
\tableofcontents
\chapter{Вступ}
Операція - сукупність дій, спрямованих на досягнення визначеної мети.
Дослідження операцій - наука, що займається дослідженням реальних процесів-операцій, виробленням рекомендацій про прийняття рішень.
Кроки при розв'язуванні задач:
\begin{enumerate}
\item Постановка задачі.
\item Побудова моделі.
\item Відшукання розв'язку.
\item Перевірка моделі та оцінка результату.
\item Впровадження розв'язку та контроль його достовірності.
\end{enumerate}
В залежності від задачі і побудованої моделі використовують різні методи, в той час, як один метод можна застосувати до різних задач. Зараз багато методів займаються лінійним програмуванням, теорією ігор і тд.

\chapter{Лінійні моделі дослідження операцій}
У випадку, якщо вихідна задача не є лінійною, то на першому етапі дослідження її або вважають лінійною, або всі залежності замінюються лінійними. Для побудови методу використовується апарат лінійного програмування.
\section{Транспортна задача}
Нехай маємо m пунктів виробництва однорідного продукту з потужностями відповідно $a_i$, $ i =\overline{1, m}$. Маємо n пунктів споживання $b_j$, $ j =\overline{1, n}$. Задається матриця перевезень С =\{ $c_{ij}$ \}, де $c_{ij}$ - вартість перевезення одиниці продукту з i-того пункту виробництва в j-ий пункт споживання. Потрібно знайти такий набір $x_{ij}$ $\geq$ 0, $ i = \overline{1, m}, j = \overline{1, n}$, де  $x_{ij}$ - кількість одиниць продукту, яка перевозиться з і-ого пункту виробництва в j-ий пункт споживання, щоб виконувались наступні умови: 
\begin{enumerate}
\item  \begin{equation}  \sum_{j=1}^n x_{ij} = a_i, i = \overline{1, m}. \end{equation}
\item   \begin{equation} \sum_{i=1}^m x_{ij} = b_j, j = \overline{1, n}.   \end{equation}
\item  \begin{equation} \sum_{i=1}^m \sum_{j=1}^n c_{ij} x_{ij} \to \min  \end{equation}
\end{enumerate}
Невідємний набір $x_{ij}$, який задовольняє (2.1), (2.2), називається планом задачі або допустимим розвязком. Той із планів, який надає мінімум в (2.3), називається оптимальним планом або розвязком транспортної задачі.
Зауважимо, що транспортна задача, поставлена в такій формі, називається транспортною задачею за критерієм вартості.
Умова \begin{equation}  \sum_{i=1}^n a_i = \sum_{j=1}^m b_j  \end{equation} називається умовою балансу мас.\\
Транспортну задачу зручно зображати таблицею:\\
\begin{tabular}{ | c | c | c | c | c | }
\hline
\slashbox{$x_{1 1}$}{$c_{1 1}$} & \slashbox{$x_{1 2}$}{$c_{1 2}$} & \vdots & \slashbox{$x_{1 n}$}{$c_{1 n}$} & $a_1$ \\
\hline
\slashbox{$x_{2 1}$}{$c_{2 1}$} & \slashbox{$x_{2 2}$}{$c_{2 2}$} & \vdots & \slashbox{$x_{2 n}$}{$c_{2 n}$} & $a_2$ \\
\hline
 $\cdots$ & $\cdots$ & $\ddots$ & $\cdots$ & $\cdots$ \\
\hline
\slashbox{$x_{m 1}$}{$c_{m 1}$} & \slashbox{$x_{m 2}$}{$c_{m 2}$} & \vdots & \slashbox{$x_{m n}$}{$c_{m n}$} & $a_m$ \\
\hline
$b_1$ & $b_2$ & \vdots & $b_n$ &  \\
\hline
\end{tabular}
\\$\Rightarrow$
\\
$\mathbf {Теорема  2}$\\
Для розвязності транспортної задачі необхідно і достатньо, щоб виконувалась умова балансу мас $\sum_{i=1}^m a_i = \sum_{j=1}^n b_j$ .
\\
$\mathbf {Доведення:}$\\
(Необхідність)\\ 
 Нехай $ x_{ij}^*, i = \overline{1, m}, j = \overline{1, n}$ , - розвязок транспортної задачі. \\
 Оскільки  $\sum_{j=1}^n x_{ij}^* = a_i,  i = \overline{1, m};  \sum_{i=1}^m  x_{ij}^* = b_j ,  j = \overline{1, n}$ , то отримаємо
 $\sum_{i=1}^n a_i = \sum_{j=1}^m b_j$.
\\
(Достатність)
\\
Нехай виконується умова (2.4), покладемо  $x_{ij} = \frac{a_ib_j}{\sum a_i}$. Сумуючи це співвідношення по j, отримаємо:\\
$\sum_{j=1}^n x_{ij} = \sum_{j=1}^n \frac{a_ib_j}{\sum a_i} = a_i,  i = \overline{1, m}.$
\\ $\sum_{j=1}^m x_{ij} = \sum_{i=1}^m \frac{a_ib_j}{\sum a_i} = b_j,  j = \overline{1, n}.$

\end{document}