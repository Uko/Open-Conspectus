% These lines tell TeXShop to typeset with xelatex, and to open and 
% save the source with Unicode encoding.

%!TEX TS-program = xelatex
%!TEX encoding = UTF-8 Unicode

\documentclass[12pt,a4paper]{book}
\usepackage{xltxtra}
\usepackage{makecell}
\usepackage[left=2cm,top=2cm,right=2cm,bottom=2cm]{geometry}
\usepackage{amsmath}
\usepackage{colortbl}
\usepackage[table]{xcolor}
\renewcommand{\chaptername}{Тема}
\setmainfont[Mapping=tex-text]{Times New Roman}

\newenvironment{slim_enumerate}{
\begin{enumerate}
  \setlength{\itemsep}{1pt}
  \setlength{\parskip}{0pt}
  \setlength{\parsep}{0pt}}
{\end{enumerate}}

\begin{document}

\tableofcontents

\chapter{Вступ}

\emph{Операція} - сукупність дій, спрямованих на досягнення визначеної мети.

\emph{Дослідження операцій} - наука, що займається дослідженням реальних процесів-операцій, виробленням рекомендацій про прийняття рішень.

Кроки при розв'язуванні задач:
\begin{slim_enumerate}
  \item Постановка задачі.
  \item Побудова моделі.
  \item Відшукання розв'язку.
  \item Перевірка моделі та оцінка результату.
  \item Впровадження розв'язку та контроль його достовірності.
\end{slim_enumerate}

В залежності від задачі і побудованої моделі використовують різні методи, в той час, як один метод можна застосувати до різних задач. Зараз багато методів займаються лінійним програмуванням, теорією ігор і тд.

\section{Алгоритм Пріма}

Нехай дано простий зв’язний зважений граф $G=(V,E)$ і вагова функція $d:E\rightarrow R$.

Потрібно знайти мінімальний каркас $A_s$ в заданому графі, починаючи з вершини $x_s$.

Алгоритм Пріма:
\begin{slim_enumerate}
  \item Нехай $T_s = \{x_s\}$ - множина вершин, з’єднаних ребрами, що входять в мінімальний каркас,\\
$A_s = \{\emptyset\}$ - множина ребер, що входять в каркас мінімальної довжини.
  \item Записати:\\
$\forall x_j\in$ Г$(x_s)$ $[\alpha_j=x_s, \beta_j=d(x_s,x_j)]$ (Г$(x_s)$ - суміжні до $x_s$ вершини)\\
$\forall x_j\notin$ Г$(x_s)$ $[0,\infty]$
  \item Вибрати $x_j^*$, де $\beta_j^*=\displaystyle\min_{x_j\notin T_s}\{\beta_j\}$,\\
$T_s=T_s\cup\{x_j^*\}$,\\
$A_s=A_s\cup\{(\alpha_j^*,x_j^*)\}$.\\
Якщо $|T_s|=n\Rightarrow$ кінець,\\
інакше ${\Rightarrow}$ Крок 4.
  \item $\forall x_j\notin T_s, x_j\in$ Г$(x_j^*), \beta_j>d(x_j^*,x_j)$ оновити мітки:\\
$\beta_j=d(x_j^*,x_j), \alpha_j=x_j^*$.\\
Перейти на Крок 3.
\end{slim_enumerate}

*місце на граф (малюнок-приклад)*

\chapter{Алгоритми пошуку шляхів}

Нехай маємо орієнтований граф $G=(V,E)$, дугам якого ставляться у відповідність ваги, що задаються матрицею $A=A_{ij}$. Ставляться такі задачі знаходження найкоротшого(х) шляху(ів):
\begin{slim_enumerate}
  \item від заданої початкової - до заданої кінцевої вершини графа;
  \item між заданою початковою вершиною графа та всіма іншими вершинами графа;
  \item між усіма парами вершин графа.
\end{slim_enumerate}
Задачі 1) і 2) розв’язують алгоритми Дейкстри (ваги $\geq 0$) і Форда (ваги довільні).

Алгоритм Дейкстри чи Форда дозволяє знайти найкоротший шлях від деякої виділеної вершини графа до будь-якої іншої. Можна було б методом багаторазового використання одного з цих алгоритмів з послідовним перебором кожної вершини графа в ролі початкової розв’язати задачу 3), але це вимагало б великої обчислюваної роботи.

Більш ефективним методом розв’язування задачі 3) є алгоритм Флойда. Для дуг допускаються від’ємні ваги, але не допускається наявність від’ємного циклу.

\section{Алгоритм Форда}

\emph{Нагадування:} ваги довільні.

Припустимо, що немає циклів з від’ємною довжиною. Навідміну від алгоритму Дейкстри, ніяка із міток під час процесу не розглядаєтсья як остаточна.

Позначимо $l^k(x_i)$ - мітка вершини $x_i$ в кінці $k-1$ операції.
\begin{slim_enumerate}
  \item \emph{Присвоєння початкових значень}\\
Нехай $x_s$ - довільна початкова вершина,\\
покласти $S=$ Г$(x_s)$,\\
$k=1, l^1(x_s)=0;$\\
$\forall x_i \in$ Г$(x_s), l^1(x_i)=d(x_s,x_i)$\\
$\forall x_i \notin$ Г$(x_s), l^1(x_i)=\infty$
  \item \emph{Оновлення міток}\\
$\forall x_i \in$ Г$(S), (x_i \neq x_s)$ знайти її мітку наступним чином:\\
$T_i=$ Г$^{-1}(x_i) \cap S$,\\
$l^{k+1}(x_i)=\min\{l^k(x_i),\displaystyle\min_{x_j \in T_i}[l^k(x_j)+d(x_j,x_i)]\}$ (тут важливий порядок - дуги),\\
$\forall x_i \notin$ Г$(S): l^{k+1}(x_i)=l^k(x_i)$
  \item \emph{Перевірка на закінчення}
    \begin{slim_enumerate}
      \item $k \leq n-1$, якщо $\forall i$ $l^{k+1}(x_i)=l^k(x_i)$, то мітки рівні довжинам найкоротших шляхів. Кінець.
      \item $k<n-1$, якщо $\exists i$ $l^{k+1}(x_i) \neq l^k(x_i)$, то перейти до Кроку 4.
      \item $k=n-1$, якщо $\exists i$ $l^{k+1}(x_i) \neq l^k(x_i)$, то в графі присутній цикл від’ємної довжини і \emph{задача не має розв’язку}. Кінець.
    \end{slim_enumerate}
  \item \emph{Підготовка до наступної ітерації}\\
Оновити мітку наступним чином:
$$S=\{x_i:l^{k+1}(x_i) \neq l^k(x_i)\}$$
  \item Покласти $k=k+1$ і перейти до Кроку 2.
\end{slim_enumerate}

Коли довжини найкоротших шляхів будуть знайдені то самі шляхи отримаємо рекурсивно:
$$l(x_i')+d(x_i',x_i)=l(x_i),$$
$x_i'$ - вершина, що безпосередньо передує $x_i$ на шляху від $x_s$.

\section{Алгоритм Флойда}

Застосовується до графів з довільними дугами, але не допускається наявність циклу від’ємної довжини.

В алгоритмі використовуються дві, оновлювані в його процесі, матриці: матриця ваг - $D$ і матриця попередніх вершин - $\Theta$. Прицьому, на $k$-й ітерації елементи матриці ваг $D_{(k)}=\{d_{ij}^{(k)}\}$ позначають найкоротший шлях між вершинами $x_i$ та $x_j$, який може складатись із внутрішніх (проміжних) вершин з множини перших $k$ вершин графу - $\{x_1, ..., x_k\}$ а елементи матриці $\Theta=\{\theta_{ij}^{(k)}\}$ позначають вершини, що безпосереньо передують вершинам $x_j$ у біжучому найкоротшому шляху від $x_i$ до $x_j$.

\emph{Внутрішня (проміжна) вершина - вершина графу, що не збігається з його початковою або кінцевою вершиною.}

\begin{slim_enumerate}
  \item \emph{Присвоєння початкових значень}\\
Нехай задано зважений граф $G=(V,E)$ i вагова функція $d:E \rightarrow R$.
$k=0$\\
$\forall (i,j) \in E d_{ij}^{(k)}=d(x_i,x_j)$ (є дуги)\\
$\forall (i,j) \notin E d_{ij}^{(k)}=\infty$ (дуги відсутні)\\
$d_{ii}^{(k)}=0$ (діагональні)\\
$\theta_{ij}^{(k)}=x_i$
  \item $k=k+1$
  \item $\forall i: i \neq k, d_{ik}^{(k-1)} \neq \infty, \forall j: j \neq k, d_{kj}^{(k-1)} \neq \infty, d_{ij}^{(k-1)}>d_{ik}^{(k-1)}+d_{kj}^{(k-1)}$ оновити матриці:\\
$d_{ij}^{(k)}=d_{ik}^{(k-1)}+d_{kj}^{(k-1)}$\\
$\theta_{ij}^{(k)}=\theta_{kj}^{(k-1)}$\\
Для всіх інших $i$ та $j$ переписати попередні елементи:\\
$d_{ij}^{(k)}=d_{ij}^{(k-1)}$,\\
$\theta_{ij}^{(k)}=\theta_{ij}^{(k-1)}$.
  \item 
    \begin{slim_enumerate}
      \item $d_{ii}^{(k)} < 0$ - в графі пристуній цикл від’ємної довжини, що містить вершину $x_i$ - розв’язку не існує. Кінець.
      \item $d_{ii}^{(k)} \geq 0, k=n$ - маємо розв’язок - матриця $D^{(n)}$ містить найкоротші шляхи між вершинами графу. Кінець.
      \item $d_{ii}^{(k)} \geq 0, k<n$ - перейти на Крок 2.
    \end{slim_enumerate}
\end{slim_enumerate}

\emph{Зауваження.} Якщо в початковій матриці $D^{(0)}$ усі діагональні елементи покласти рівними $\infty$ то $d_{ii}^{(n)}$ буде рівним вазі ланцюга що проходить через $x_i$.

\chapter{Лінійні моделі дослідження операцій}

У випадку, якщо вихідна задача не є лінійною, то на першому етапі дослідження її або вважають лінійною, або всі залежності замінюються лінійними. Для побудови методу використовується апарат лінійного програмування.

\section{Транспортна задача}

Нехай маємо $m$ пунктів виробництва однорідного продукту (бази, склади, ...) з потужностями, відповідно, $a_i, i = \overline{1, m}$. Маємо $n$ пунктів споживання, відповідно, з потребами $b_j, j =\overline{1, n}$.

Задається матриця перевезень $С = \{c_{ij}\}$, де $c_{ij}$ - вартість перевезення одиниці продукту з $i$-того пункту виробництва в $j$-й пункт споживання.

Потрібно знайти такий набір $x_{ij} \geq 0, i = \overline{1, m}, j = \overline{1, n}$, де  $x_{ij}$ - кількість одиниць продукту, яка перевозиться з $і$-го пункту виробництва в $j$-й пункт споживання, щоб виконувались наступні умови: 
\begin{equation}  \sum_{j=1}^n x_{ij} = a_i, i = \overline{1, m}. \end{equation}
\begin{equation} \sum_{i=1}^m x_{ij} = b_j, j = \overline{1, n}.   \end{equation}
\begin{equation} \sum_{i=1}^m \sum_{j=1}^n c_{ij} x_{ij} \to \min  \end{equation}

Невідємний набір $x_{ij}$, який задовольняє (3.1), (3.2), називається \emph{планом задачі} або \emph{допустимим розвязком}. Той із планів, який надає мінімум в (3.3), називається \emph{оптимальним планом} або розвязком транспортної задачі.

\emph{Зауважимо}, що транспортна задача, поставлена в такій формі, називається "транспортною задачею за критерієм вартості".

Умова \begin{equation}  \sum_{i=1}^n a_i = \sum_{j=1}^m b_j  \end{equation} називається умовою \emph{балансу мас}.

Транспортну задачу зручно зображати таблицею:\\
\begin{tabular}{ | c | c | c | c | c | }
\hline
\diaghead(4,3){easterr}{$c_{1 2}$}{$x_{1 2}$} & \diaghead(4,3){easterr}{$c_{1 2}$}{$x_{1 2}$} & \thead{\vdots} & \diaghead(4,3){easterr}{$c_{1 n}$}{$x_{1 n}$} & \thead{$a_1$} \\
\hline
\diaghead(4,3){easterr}{$c_{2 1}$}{$x_{2 1}$} & \diaghead(4,3){easterr}{$c_{2 2}$}{$x_{2 2}$} & \thead{\vdots} & \diaghead(4,3){easterr}{$c_{2 n}$}{$x_{2 n}$} & \thead{$a_2$} \\
\hline
 \thead{$\cdots$} & \thead{$\cdots$} & \thead{$\ddots$} & \thead{$\cdots$} & \thead{$\cdots$} \\
\hline
\diaghead(4,3){easterr}{$c_{m 1}$}{$x_{m 1}$} & \diaghead(4,3){easterr}{$c_{m 2}$}{$x_{m 2}$} & \thead{\vdots} & \diaghead(4,3){easterr}{$c_{m n}$}{$x_{m n}$} & \thead{$a_m$} \\
\hline
\thead{$b_1$} & \thead{$b_2$} & \thead{\vdots} & \thead{$b_n$} & \thead{} \\
\hline
\end{tabular}

Матричний вигляд умов (3.1), (3.2) - матриця обмежень:\\
\begin{tabular}{ @{\hspace{1.4em}}l l }
$
\setlength{\arraycolsep}{0.27em}
\begin{array}{ccccccccccccc}
A_{1 1} & A_{1 2} & \dots & A_{1 n} & A_{2 1} & A_{2 2} & \dots & A_{2 n} & \dots & A_{m 1} & A_{m 2} & \dots & A_{m n} 
\end{array}$ &  \\
\multicolumn{2}{l}{
$\left(
 \begin{array}{ccccccccccccc}
1 & 1 & \dots & 1 & 0 & 0 & \dots & 0 & \dots & 0 & 0 & \dots & 0 \\
0 & 0 & \dots & 0 & 1 & 1 & \dots & 1 & \dots & 0 & 0 & \dots & 0 \\
\dots & \dots & \dots & \dots & \dots & \dots & \dots & \dots & \dots & \dots & \dots & \dots & \dots \\
0 & 0 & \dots & 0 & 0 & 0 & \dots & 0 & \dots & 1 & 1 & \dots & 1 \\
1 & 0 & \dots & 0 & 1 & 0 & \dots & 0 & \dots & 1 & 0 & \dots & 0 \\
\dots & \dots & \dots & \dots & \dots & \dots & \dots & \dots & \dots & \dots & \dots & \dots & \dots \\
0 & 0 & \dots & 1 & 0 & 0 & \dots & 1 & \dots & 0 & 0 & \dots & 1
\end{array}\right)
\left(\begin{array}{c}
x_{1 1} \\
x_{1 2} \\
\dots \\
x_{1 n} \\
x_{m 1} \\
\dots \\
x_{m n}
\end{array}
\right)
=
\left(\begin{array}{c}
a_1 \\
a_2 \\
\dots \\
a_m \\
b_1 \\
\dots \\
b_n
\end{array}
\right)$}
\end{tabular}

\subsection{Теорема 1}

Ранг матриці обмежень транспортної задачі рівний $r=m+n-1$.

{\bf Доведення:}

Оскільки сума перших рівнянь (3.1) рівна сумі наступних $n$ рівнянь (3.2), тобто виконується умова (3.4), то звідси випливає, що $r \leq m+n-1$.

Для того, щоб показати, що $r=m+n-1$ виділимо в матриці обмежень квадратну матрицю розмірності $m+n-1$ визначник якої нерівний нулю. Для цього можна взяти вектори:\\
\begin{tabular}{ @{\hspace{1em}}l l }
$
\setlength{\arraycolsep}{0.23em}
\begin{array}{cccccccc}
A_{1 n} & A_{2 n} & \dots & A_{m n} &  A_{1 1} & A_{1 2} & \dots & A_{1 n-1} 
\end{array}$ &  \\
\multicolumn{2}{l}{
$\left|
 \begin{array}{cccccccc}
1 & 0 & \dots & 0 & 1 & 1 & \dots & 1 \\
0 & 1 & \dots & 0 & 0 & 0 & \dots & 0 \\
\dots & \dots & \dots & \dots & \dots & \dots & \dots & \dots \\
0 & 0 & \dots & 1 & 0 & 0 & \dots & 0 \\
0 & 0 & \dots & 0 & 1 & 0 & \dots & 0 \\
0 & 0 & \dots & 0 & 0 & 1 & \dots & 0 \\
\dots & \dots & \dots & \dots & \dots & \dots & \dots & \dots \\
0 & 0 & \dots & 0 & 0 & 0 & \dots & 1 
\end{array}\right|$}
\end{tabular}
$\neq 0$

\subsection{Теорема  2}

Для розвязності транспортної задачі необхідно і достатньо, щоб виконувалась умова балансу мас $\sum_{i=1}^m a_i = \sum_{j=1}^n b_j$ .

{\bf Доведення:}

{\it Необхідність.} Нехай $ x_{ij}^*, i = \overline{1, m}, j = \overline{1, n}$ , - розвязок транспортної задачі.

Оскільки  $\sum_{j=1}^n x_{ij}^* = a_i,  i = \overline{1, m};  \sum_{i=1}^m  x_{ij}^* = b_j ,  j = \overline{1, n}$ , то отримаємо $\sum_{i=1}^n a_i = \sum_{j=1}^m b_j$.

{\it Достатність.} Нехай виконується умова балансу мас, покладемо  $x_{ij} = \frac{a_ib_j}{\sum_{i=1}^m a_i}$. Сумуючи це співвідношення по $j$, отримаємо:

$\sum_{j=1}^n x_{ij} = \sum_{j=1}^n \frac{a_ib_j}{\sum_{i=1}^m a_i} = a_i,  i = \overline{1, m}.$

$\sum_{i=1}^m x_{ij} = \sum_{i=1}^m \frac{a_ib_j}{\sum_{i=1}^m a_i} = b_j,  j = \overline{1, n}.$

\section{Властивості опорних планів транспортної задачі}

План $\{x_{ij}\}_{m,n}$ транспортної задачі називають \emph{опорним планом}, якщо вектори $A_{ij}$ (з матриці обмежень), що відповідають додатним компонентам плану лінійно незалежні.

Оскільки із Т1 $\Rightarrow$ ранг матриці обмедень транспортної задачі $r=m+n-1$, то додатних компонент опорний план може мати не більше ніж $m+n-1$.

\begin{itemize}
  \item Опорний план, який має рівно $m+n-1$ додатних компонент - невироджений.
  \item Опорний план, який має менше $m+n-1$ додатних компонент - вироджений.
\end{itemize}

\emph{Базисом ОП} називається довільна система із $m+n-1$ лінійно незалежних векторів $A_{ij}$, яка містить усі вектори $A_{ij}$, що відповідають додатним компонентам плану.

Поставимо взаємовідповідність між клітинами транспонованої таблиці і векторами $A_{ij}$. Кожній клітині $(i,j) \leftrightarrow A_{ij}$.

Набір клітин $$(i_1,j_1),(i_1,j_2),(i_2,j_2),\dots,(i_s,j_1)$$або$$(i_1,j_1),(i_2,j_1),(i_2,j_2),\dots,(i_1,j_s)$$
називають \emph{ланцюжком}. Звідси видно, що два сусідні елементи ланцюжка лежать або в одному рядку або в одному стовпчику.

\emph{Зауважимо}, що кількість елементів замкненого ланцюжка завжди парна.

\dots табличка з прикладом ланцюжка \dots

Нехай $P$ - довільна система векторів $A_{ij}$ умов транспортної задачі, $I$ - множина пар індексів $(i,j)$, які відповідають векторам $A_{ij} \in P$.

\subsection{Теорема 3}

Для того, щоб система векторів $P$ була лінійно незалежною необхідно і достатньо, щоб із елементів множини $I$ неможна було скласти замкнений ланцюжок.

{\bf Доведення:}

{\it Необхідність.} $P$ - лінійно незалежна система, покажем, що неможна замкнути ланцюжок. {\it Від супротивного.} Припустимо, що із елементів множини $I$ можна скласти замкнений ланцюжок: $$(i_1,j_1), (i_1,j_2), (i_2,j_2), \dots, (i_s,j_1).$$ Звідси випливає, враховуючи вигляд векторів $A_i$, $$A_{{i_1},{j_1}}-A_{{i_1},{j_2}}+A_{{i_2},{j_2}}-\dots-A_{{i_s},{j_1}}=0,$$ а тому система векторів лінійно залежна, що суперечить вхідній умові.

{\it Достатність.} Припустимо, що замкнений ланцюжок не скласти. Покажемо, що система векторів лінійно незалежна. {\it Від супротивного.} Нехай вектори лінійно залежні. Звідси випливає, що $\exists \alpha_{ij} \neq 0, (i,j) \in I:$ $$\sum_{(i,j) \in I}\alpha_{ij}A_{ij} = 0.$$

Нехай $\alpha_{{i_1}{j_1}} \neq 0$, тоді: 
$$\sum_{(i,j) \in I}\alpha_{ij}A_{ij} = -\alpha_{{i_1}{j_1}}A_{{i_1}{j_1}};$$
$$I_1 = I\setminus\{(i_1,j_1)\}.$$

Компонента $i_1$ вектора в правій частині не рівна нулю, тому в лівій частині існує принаймі один вектор $A_{{i_1}{j_2}}: \alpha_{{i_1}{j_2}}\neq0$, тоді 
$$\sum_{(i,j){\in}I}\alpha_{ij}A_{ij} = -\alpha_{{i_1}{j_1}}A_{{i_1}{j_1}}-\alpha_{{i_1}{j_2}}A_{{i_1}{j_2}}.$$

Оскільки $j_1 \neq j_2$ і $m + j_2$ компонента парвої частини не рівна нулю, то знайдеться принаймі один вектор $A_{{i_2}{j_2}}: \alpha_{{i_2}{j_2}} \neq 0$, тоді $$\sum_{(i,j) \in I}\alpha_{ij}A_{ij} = -\alpha_{{i_1}{j_1}}A_{{i_1}{j_1}}-\alpha_{{i_1}{j_2}}A_{{i_1}{j_2}}-\alpha_{{i_2}{j_2}}A_{{i_2}{j_2}}.$$
і т.д.

Цей процес скінченний, оскільки всі вектори в лівій частині різні, то врезультаті приходимо до:
$$0 = -\alpha_{{i_1}{j_1}}A_{{i_1}{j_1}}-\alpha_{{i_2}{j_1}}A_{{i_2}{j_1}}-\dots-\alpha_{{i_k}{j_k-1}}A_{{i_k}{j_k-1}}, i_k=i_s, 1 \leq s \leq k-2;$$
$$0 = -\alpha_{{i_1}{j_1}}A_{{i_1}{j_1}}-\alpha_{{i_1}{j_2}}A_{{i_1}{j_2}}-\dots-\alpha_{{i_k}{j_{k+1}}}A_{{i_k}{j_{k+1}}}, j_{k+1}=j_l, l \leq 1 \leq k-1.$$

Тоді із елементів $(i_1,j_1), (i_1,j_2), \dots, (i_k,j_k)$ можна скласти замкнений ланцюг:
$$(i_s,j_s), (i_{s+1},j_s), \dots, (i_k = i_s, j_{k-1})$$
$$(i_l,j_l), (i_l,j_{l+1}), \dots, (i_k, j_{k+1} = j_l),$$
що суперечить нашому припущенню.

\section{Методи побудови початкових опорних планів транспортної задачі}
\subsection{Метод Північно-Західного кута}
Припустимо, що ТЗ задана таблично:\\
\begin{tabular}{ | c | c | c | c | c |}
\hline
$x_{11}$/$c_{11}$	&	/$c_{12}$	&	\dots	&	/$c_{1n}$	&	$a_1$\\
\hline
$x_{21}$/$c_{21}$	&	/$c_{22}$	&	\dots	&	/$c_{2n}$	&	$a_2$\\
\hline
\dots	&	\dots	&	\dots	&	\dots	&	\dots\\
\hline
$x_{m1}$/$c_{m1}$	&	/$c_{m2}$	&	\dots	&	/$c_{mn}$	&	$a_m$\\
\hline
$b_1$	&	$b_2$	&	\dots	&	$b_n$	&\\
\hline
\end{tabular}\\
Заповнюєм транспортну таблицю починаючи з крайнього верхнього і лівого кута.\\
$$x_{11} = min\{a_1,b_1\}=a_1.$$
Це означає, що запас першого пункту вичерпаний із наступного розподілу цей пункт виключаємо, а потреби першого пункту споживання рівні $b'_1=b_1-a_1$.
$$x_{21} = min\{a_2,b'_1\} = min\{a_2,b_1-a_1\} = b'_1.$$
$$a'_2=a_2-b'_1=a_2-(b_1-a_1).$$
При заповненні останньої клітини, оскільки в нас виконуєтсья умова балансу мас, можливості і потреби повинні бути рівними.\\
Із попередньої теореми випливає, що план побудований таким методом є опорним.\\
\begin{tabular}{ | c | c | c | c | c |}
\hline
8/3	&	1/5	&	/4	&	/7		&	9\\
\hline
/7	&	9/8	&	6/9	&	/11		&	15\\
\hline
/4	&	/6	&	13/8	&	10/14	&	23\\
\hline
8	&	10	&	19	&	10		&\\
\hline
\end{tabular}
Отримали невироджений ОП, бо кількість заповнених клітин рівна $m+n-1 = 3+4-1 = 6$.\\
\subsection{Метод мінімального елементу}
При побудові початкового ОП методом Пн.-Зх. кута ми не враховували елементів матриці вартости. Природно сподіватися, що коли враховувати елементи матриці вартостей, то отримаємо ОП кращий від попереднього, тобто затрати на перевезення будуть меншими. Таким методом є метод мінімального елементу.\\
Серед елементів матриці вартостей шукаємо мінімальний. Припустимо це елемент $c_{kl}$. Заповнення таблиці починаємо із клітини $(k,l)$, аналогічно як в попередньому методі.
$$x_{kl} = min\{a_k,b_l\},$$
при цьому, або рядок, або стовпчик із наступного розгляду викреслюємо. Відповідно змінюємо або запас в $k-$му пункті постачання, або потреби $l-$го пункту споживання.\\
В матриці, що залишається знову шукаєм мінімальний елемент і т.д.\\
\begin{tabular}{ | c | c | c | c | c |}
\hline
8/3	&	/5	&	1/4	&	/7		&	9\\
\hline
/7	&	/8	&	5/9	&	10/11	&	15\\
\hline
/4	&	10/6	&	13/8	&	/14		&	23\\
\hline
8	&	10	&	19	&	10		&\\
\hline
\end{tabular}
Перевіримо чи отриманий ОП кращий попереднього:\\
далі буде\dots\\
\\
\\
\subsection{Метод Фогеля}
В кожному рядку шукаємо мінімальний елемент і наступний за ним по величині. Різницю записуємо справа від рядка. Аналогічно поступаєво із стовпчиками, записуємо внизу кожного.\\
Серед отриманих різниць шукаємо максимальну і в стовпчику чи рядку, якому віповідає максимальна різниця шукаємо мінімальний елемент матриці вартості.\\
заповнення транспортної таблиці починаємо з отриманої клітини аналогічно як і в попередніх методах.\\
\begin{tabular}{ | c | c | c | c | c |}
\hline
/3	&	/5	&	9/4	&	/7		&	9\\
\hline
/7	&	/8	&	5/9	&	10/11	&	15\\
\hline
8/4	&	10/6	&	5/8	&	/14		&	23\\
\hline
8	&	10	&	19	&	10		&\\
\hline
\end{tabular}

\section{Двоїста ТЗ}
\section{Умови оптимальності ЕЗ}
\subsection{Теорема}
\dots
$$u_i+v_j=c_{ij},	x_{ij}>0\;\;\;(7)$$
$$u_i+v_j{\leq}c_{ij},	x_{ij}=0\;\;\;(8)$$
\dots
\subsection{Метод потенціалів (розв’язування ТЗ)}
(Базується на попередній теоремі).\\
Попередній крок\\
Яким-небудь із методів, наприклад методом Пн-Зх кута будуємо початковий ОП ТЗ.\\
\\
Припускаємо, що ОП - невироджений.\\
\\
Загальний крок\\
Використовуючи, клітини в яких стоять додатні перевезення і які будемо називати "заповненими", використовуючи умову $(7)$, визначаємо числа-потенціали: $u_i, v_j$. Враховуючи, що невідомих є $n+m$, а заповнених клітин $n+m-1$  (ОП - невироджений), то будь-якій із змінних, для зручності $u_1$, присвоюємо 0 і знаходимо всі інші.\\
Далі для незаповнених клітин, які називатимемо "вільними", перевіряєм умову $(8)$:
$$\gamma_{ij}=c_{ij}-u_i-v_j{\geq}0\;\;\;(8')$$
Якщо умова $(8')$ виконується для всіх вільних клітин, то маємо оптимальний план, якщо ні - то отриманий план будемо покращувати.\\
Для цього взначимо клітину із найбільшим порушенням умови оптимальності, нехай це клітина $(k,l)$. Зауважимо, що можна було б й надалі використовувати першу клітину з порушенням умови оптимальності, яка зустрілась, але використання клітини із найбільшим порушенням умови оптимальності дає можливість визначити оптимальний план за менше число кроків.
$$\gamma_{kl}=min\{\gamma_{ij}\} - клітина (k,l) з найбільшим порушенням$$
Отже для вільної клітини $(k,l)$ будуємо ланцюжок (замкнутий контур), використовуючи заповнені клітини, який замикаєтсья на клітині $(k,l)$. (Використовуємо стільки, скільки потрібно).\\
Тоді для клітин ланцюга присвоюємо знаки, чергуючи $+,-$, причому клітині $(k,l)$ присвоюємо "$+$". Серед клітин, які ввійшли із знаком "$-$" шукамо мінімальне перевезення і віднімаємо від усіх перевезень в клітинах з "$-$" і додаємо до перевезень в клітини з "$+$". Отриманий план перевірямо на оптимальність, тобто повторюємо загальний крок.
\subsection{Метод потенціалів. Обґрунтування зменшення знаку лінійної форми}
\section{Відкрита та закрита модель ТЗ}
\section{Задача про призначення}
\subsection{Теорема 1}
\subsection{Теорема 2}
\subsection{Алгоритм}
\section{Метод диференціальних рент (розв’язування ТЗ)}
Зручний тим, що в процесі розв’язування ми не зустрічаємся з випадком виродженості.\\
Вважається що ТЗ задана таблицею.\\
\begin{tabular}{ | c | c | c | c | c | c | }
\hline
	&		&		&		&		&\\
\hline
	&	/2	&	/4	&	/5	&	/1	&	60\\
\hline
	&	/2	&	/3	&	/9	&	/4	&	70\\
\hline
	&	/3	&	/4	&	/2	&	/5	&	20\\
\hline
	&	40	&	30	&	30	&	50	&\\
\hline
\end{tabular}\\
В кожному стопці шукаємо мінімальну вартість перевезення і беремо її в кільце.\\
\begin{tabular}{ | c | c | c | c | c | c | }
\hline
	&		&		&		&		&\\
\hline
	&	/(2)	&	/4	&	/5	&	/(1)	&	60\\
\hline
	&	/(2)	&	/(3)	&	/9	&	/4	&	70\\
\hline
	&	/3	&	/4	&	/(2)	&	/5	&	20\\
\hline
	&	40	&	30	&	30	&	50	&\\
\hline
\end{tabular}\\
Після цього здійснюємо розподіл, причому, заповнюємо тільки ті клітини, які відмічені кільцями.\\
(Порівнюємо обсяг виробництва по відповідному рядку і обсяг споживання по відповідному стовпчику. Менший із порівнюваних обсягів приймаємо ща величину перевезення. Перевіряємо чи весь обсяг виробництва розподілений.)\\
\begin{tabular}{ | c | c | c | c | c | c | }
\hline
	&		&		&		&		&\\
\hline
	&	40/(2)	&	/4	&	/5	&	20/(1)	&	60\\
\hline
	&	/(2)	&	30/(3)	&	/9	&	/4	&	70\\
\hline
	&	/3	&	/4	&	20/(2)	&	/5	&	20\\
\hline
	&	40	&	30	&	30	&	50	&\\
\hline
\end{tabular}\\
Якщо запас пункту виробництва вичерпаний, а потреби пунктів споживання зв’язаних кільцем з цим пунктом виробництва є повністю задоволені, то пункт виробництва вважається недостатнім, а рядок від’ємним.\\
Якщо ж в пукнті виробництва єє нерозподілений залишок , всі пункти споживання, пов’язані кільцем із цим пунктом виробництва, є повністю задоволені то пункт вважається надлишковим, а рядок додатним.\\
\begin{tabular}{ | c | c | c | c | c | c | }
\hline
	&		&		&		&		&\\
\hline
-30	&	40/(2)	&	/4	&	/5	&	20/(1)	&	60\\
\hline
+40	&	/(2)	&	30/(3)	&	/9	&	/4	&	70\\
\hline
-10	&	/3	&	/4	&	20/(2)	&	/5	&	20\\
\hline
	&	40	&	30	&	30	&	50	&\\
\hline
\end{tabular}\\
Якщо не розподілений залишок по рядку є $0$, то якщо цей рядок кільцем пов’язаний із додатнім рядком, то знак "$+$", якщо з від’ємним - "$-$". Якщо одночасно пов’язаний із від’ємним і додатнім то у відповідному пункті виробництва збільшуємо обсяг виробництва і здійснюємо після цього розподіл. Якщо обсяг поставок, по цьому рядку збільшиться то ставим "$-$", якщо не - "$+$".\\
Надлишок чи недостачу у кожному рядку записуємо зліва у рядку з відповідним знаком.\\
Наступний етап розв’язування полягає у визначення різниць між найменшою вартістю у додатньому рядку і вартістю у кільці. Ці числа записуємо в додатковому рядку на відповідними стовпчиками.\\
Зауважимо: якщо кільце стоїть у додатньому рядку, то різниця не обчислюється.\\
Серед отриманих різниць шукаєм мінімальну, яка називається проміжна рента і позначається $d$.\\
\begin{tabular}{ | c | c | c | c | c | c | }
\hline
	&	-	&	-	&	7	&	3	&	$d=3$\\
\hline
-30	&	40/(2)	&	/4	&	/5	&	20/(1)	&	60\\
\hline
+40	&	/(2)	&	30/(3)	&	/9	&	/4	&	70\\
\hline
-10	&	/3	&	/4	&	20/(2)	&	/5	&	20\\
\hline
	&	40	&	30	&	30	&	50	&\\
\hline
\end{tabular}\\
Переходимо до наступної таблиці. До вартостей перевезень у від’ємних рядках додаємо величину проміжної ренти,а вартості у додатніх - не змінюємо.\\
\begin{tabular}{ | c | c | c | c | c | c | }
\hline
	&		&		&		&		&\\
\hline
	&	/5	&	/7	&	/8	&	/4	&	60\\
\hline
	&	/2	&	/3	&	/9	&	/4	&	70\\
\hline
	&	/6	&	/7	&	/5	&	/8	&	20\\
\hline
	&	40	&	30	&	30	&	50	&\\
\hline
\end{tabular}\\
Знову в кожному стовпці шукаємо мінімальну вартість.\\
Зауважимо: в стовпчику, якому відповідала проміжна рента з’являється ще один мінімальний елемент.\\
Зауважимо: ще мінімальний елемент, в деяких випадках, може появитись в стовпчиках, яким відповідає проміжна рента.\\
\begin{tabular}{ | c | c | c | c | c | c | }
\hline
	&		&		&		&		&\\
\hline
	&	/5	&	/7	&	/8	&	/(4)	&	60\\
\hline
	&	/(2)	&	/(3)	&	/9	&	/(4)	&	70\\
\hline
	&	/6	&	/7	&	/(5)	&	/8	&	20\\
\hline
	&	40	&	30	&	30	&	50	&\\
\hline
\end{tabular}\\
Після цього здійснюється розподіл обсягу вироництв. Оскільки тепер більше елементів в кільці ніж рядків, то розподіл здійснюється по іншому.\\
Починаємо перегляд по рядках або по стовпчиках, при цьому заповнюємо клітину з кільцем лише тоді, коли вона є єдиною відміченою клітиною в своєму рядку, якщо перегляд по рядках і в стовпці - якщо по стовпцях. При повторному перегляді заповнена клітина вже не враховується. \\
(Спочатку по рядках)
\begin{tabular}{ | c | c | c | c | c | c | }
\hline
	&		&		&		&		&\\
\hline
	&	/5	&	/7	&	/8	&	50/(4)	&	60\\
\hline
	&	/(2)	&	/(3)	&	/9	&	/(4)	&	70\\
\hline
	&	/6	&	/7	&	20/(5)	&	/8	&	20\\
\hline
	&	40	&	30	&	30	&	50	&\\
\hline
\end{tabular}\\
(Потім по стовпцях)
\begin{tabular}{ | c | c | c | c | c | c | }
\hline
	&		&		&		&		&\\
\hline
	&	/5	&	/7	&	/8	&	50/(4)	&	60\\
\hline
	&	40/(2)	&	30/(3)	&	/9	&	/(4)	&	70\\
\hline
	&	/6	&	/7	&	20/(5)	&	/8	&	20\\
\hline
	&	40	&	30	&	30	&	50	&\\
\hline
\end{tabular}\\
Знову обчилюєм в кожному рядку недстачу чи надлишок, присвоюєм відповідні знаки, обсилюєм величину проміжної ренти d і переходим до наступної таблиці. Продовжуємо до того часу, поки нерозподілений залишок стане рівний нулю.\\
\begin{tabular}{ | c | c | c | c | c | c | }
\hline
	&	-	&	-	&	3	&	-	&	$d=3$\\
\hline
+10	&	/5	&	/7	&	/8	&	50/(4)	&	60\\
\hline
+0	&	40/(2)	&	30/(3)	&	/9	&	/(4)	&	70\\
\hline
-10	&	/6	&	/7	&	20/(5)	&	/8	&	20\\
\hline
	&	40	&	30	&	30	&	50	&\\
\hline
\end{tabular}
$\rightarrow$
\begin{tabular}{ | c | c | c | c | c | c | }
\hline
	&		&		&		&		&\\
\hline
0	&	/5	&	/7	&	/8	&	50/(4)	&	60\\
\hline
0	&	40/(2)	&	30/(3)	&	/9	&	/(4)	&	70\\
\hline
0	&	/9	&	/10	&	20/(8)	&	/11	&	20\\
\hline
	&	40	&	30	&	30	&	50	&\\
\hline
\end{tabular}\\
\section{Метод розв’язування ТЗ за критерієм часу}
\begin{tabular}{ | c | c | c | c | c | }
\hline
$11/1^-$	&	7/3	&	$/7^+$	&\cellcolor[rgb]{0.8,0.8,0.8}	/12	&	18\\
\hline
$/2^+$	&	2/8	&	$8/10^-$	&	/8	&	10\\
\hline
/6	&	/1	&	5/4	&	7/5	&	12\\
\hline
11	&	9	&	13	&	7	&\\
\hline
\end{tabular}
$t'=10$\\
Загальний крок.\\
Визначимо $t'=\displaystyle\max_{x_{ij}>0}\{t_{ij}\}$.\\
Закреслюємо всі вільні клітини, для яких $t_{ij}{\geq}t'$.\\
Будемо покращувати план $x^0$. Стараємося перевезення $x'_{ij}$, що відповідають $t'$, по можливості, зробити нулем. Це ає змогу отримати оптимальнйи план за меншу кількість кроків.\\
Для заповнення клітини, відповідає $t'$ будуємо ланцюжок (замкнутий контур) починаючи із цієї клітини, якій присвоюємо знак "$-$", при цьому, від’ємними беремо клітини з $x_{ij}>0$, а додатними з $t_{ij}<t'$.\\
Зауважимо: це відбувається так як і в методі потенціалів. Серед від’ємних клітин шукаємо мінімальне перевезення і віднімаєм від від’ємних і додаєм до додатніх. Переходим до наступної транспортної таблички.\\
\begin{tabular}{ | c | c | c | c | c | }
\hline
$3/1^-$	&	$7/3^+$	&	8/7	&\cellcolor[rgb]{0.8,0.8,0.8}	/12	&	18\\
\hline
$8/2^+$	&	$2/8^-$	&\cellcolor[rgb]{0.8,0.8,0.8}	/10	&\cellcolor[rgb]{0.8,0.8,0.8}	/8	&	10\\
\hline
/6	&	/1	&	5/4	&	7/5	&	12\\
\hline
11	&	9	&	13	&	7	&\\
\hline
\end{tabular}
$t'=8$\\
Повторюємо загальний крок до того часу, поки не можливо буде побудувати потрібний ланцюжок.\\
\begin{tabular}{ | c | c | c | c | c | }
\hline
1/1	&	9/3	&	$8/7^-$	&\cellcolor[rgb]{0.8,0.8,0.8}	/12	&	18\\
\hline
10/2	&\cellcolor[rgb]{0.8,0.8,0.8}	/8	&\cellcolor[rgb]{0.8,0.8,0.8}	/10	&\cellcolor[rgb]{0.8,0.8,0.8}	/8	&	10\\
\hline
/6	&	/1	&	5/4	&	7/5	&	12\\
\hline
11	&	9	&	13	&	7	&\\
\hline
\end{tabular}\\
Оскільки ми не можем побудувати ланцюжок - то $t^*=7$, а план $x_{11},x_{12},x_{13},x_{21},x_{33},x_{34}$ - оптимальний.\\
\section{Задача про максимальний потік}
Нехай на площині маємо n+2 точки P0,P1,...,Pn+1, при цьому деякі впорядковані пари Pi,Pj з’єднані ланкою (Pi,Pj) так, що утворюютть зв’язний ланцюг (сітку).\\
\\
\\
*місце на малюнок*\\
\\
\\
Ланки (Pi,Pj), (Pj,Pi) - симетричні.
По шляхах mu(P0,Pi1,Pi2,...,Pn,Pn+1), що складаються із ланок (P0,Pi1), (Pi1,Pi2), ..., (Pn,Pn+1) і не утворюють петель рідина, газ або транспорт із точки P0 - входу сітки, поступає в Pn+1 - вихід сітки.\\
Кожній ланці (Pi,Pj) ставиться у відповідність число aij>=0, яке називають пропускною здатністю ланки і означає кількість речовини, яку може пропустити ця ланка за одиницю часу.\\
Потоком xij по ланці (Pi,Pj) називають кількість речовини, яка проходить по цій ланці за одиницю часу.\\
Логічно, що потоки задовольняють умовам:\\
0<=xij<=aij, i,j=0,1,..,n+1; (1)\\
summ k=0 to n xki = summ k=1 to n+1 xik, i=1,..,n. (2)\\
Умова (1)означає, що величина потому в ланці на може перевищувати пропускної здатності цієї ланки.\\
Умова (2) означає, що кількість речовини, яка поступає в будь-яку точку сітки, крім P0 -входу і Pn+1 - виходу сітки, збігається із кількістю речовини, яка виходить із цієї точки.\\
Із (2) => summ i=1 to n+1 x0i = summ i=0 to n xin+1 = Z (3)\\
Лінійну форму Z називають величиною потоку в сітці. Ставиться задача знайти величину масимального потоку в сітці, тобто знайти такий набір xij*, який задовольняє (1),(2) і максимізує лінійну форму (3).\\
Зауважимо, що задача (1-3) є задачою лінійного програмування і, взагалі кажучи, можу бути розв’язана симплекс методом.\\
Перш ніж розглядати більш ефективний алгоритм, введемо деякі попередній поняття.\\
Всі точки сітки розіб’ємо на дві множини U,V, які не перетинаються і P0єU, Pn+1єV.\\
Розглянемо множину ланок, що виходять з U і входять у V, і назвемо її перерізом сітки, позначимо (U,V).\\
Пропускною здатністю перерізу називають величину\\
(4) A(U,V) = summ on PiєU,PjєV aij;\\
Очевидно, що для довільного потоку Z і довільного перерізу (U,V) виконується\\
(5) Z<=A(U,V).\\
Переріз із найменшою пропускною здатністю називають мінімальний перерізом.\\
\subsection{Алгоритм}
Умови задачі представимо таблицею.\\
\begin{tabular}{ | c | c | c | c | c | c | c | c | }
\hline
	&	P0	&	...	&	Pi	&	...	&	Pj	&	...	&	Pn+1\\
\hline
P0	&		&		&	a0i	&		&	a0j	&		&	a0n+1\\
\hline
...	&		&		&		&		&		&		&\\
\hline
Pi	&	ai0	&		&		&		&		&		&	ain+1\\
\hline
...	&		&		&		&		&		&		&\\
\hline
Pj	&	aj0	&		&		&		&		&		&	ajn+1\\
\hline
...	&		&		&		&		&		&		&\\
\hline
Pn+1	&	an+10	&		&	an+1i	&		&	an+1j	&		&\\
\hline
\end{tabular}\\
В клітину (Pi,Pj) проставляємо пропускні здатності aij>0, зауважимо, якщо aij=0, то цей 0 записуємо в таблицю, а якщо відсутня ланка між Pl,Pm - то alm=0, але цей 0 не заповнюється і всі aii=0 теж не заповнюються.\\
Загальний k-й крок.\\
п.1 Відшукання шляху 
\section{Задача комівояжера}
\chapter{Основи мережевого планування}
\section{Часові параметри детермінованих сіток (мереж)}
\section{Ранні та пізні строки початку та закінчення робіт}
\section{Резерви часу шляхів та робіт}
\chapter{Теорія ігор}
\section{Класифікація ігор}
\section{Максимільні теореми}
\subsection{Теорема 1}
\subsection{Теорема 2}
\section{*Шось про мішані стратегії*}
\subsection{Теорема 1}
\section{Властивості оптимальних мішаних стратегій}
\subsection{Теорема 2}
\subsection{Наслідок}
\subsection{Теорема 3}
\section{Спрощення ігор}
\subsection{Теорема 4}
\subsection{Теорема 5}
\section{Розв’язування матричних ігор 2х2}
\end{document}